% Options for packages loaded elsewhere
\PassOptionsToPackage{unicode}{hyperref}
\PassOptionsToPackage{hyphens}{url}
%
\documentclass[
]{article}
\usepackage{lmodern}
\usepackage{amssymb,amsmath}
\usepackage{ifxetex,ifluatex}
\ifnum 0\ifxetex 1\fi\ifluatex 1\fi=0 % if pdftex
  \usepackage[T1]{fontenc}
  \usepackage[utf8]{inputenc}
  \usepackage{textcomp} % provide euro and other symbols
\else % if luatex or xetex
  \usepackage{unicode-math}
  \defaultfontfeatures{Scale=MatchLowercase}
  \defaultfontfeatures[\rmfamily]{Ligatures=TeX,Scale=1}
\fi
% Use upquote if available, for straight quotes in verbatim environments
\IfFileExists{upquote.sty}{\usepackage{upquote}}{}
\IfFileExists{microtype.sty}{% use microtype if available
  \usepackage[]{microtype}
  \UseMicrotypeSet[protrusion]{basicmath} % disable protrusion for tt fonts
}{}
\makeatletter
\@ifundefined{KOMAClassName}{% if non-KOMA class
  \IfFileExists{parskip.sty}{%
    \usepackage{parskip}
  }{% else
    \setlength{\parindent}{0pt}
    \setlength{\parskip}{6pt plus 2pt minus 1pt}}
}{% if KOMA class
  \KOMAoptions{parskip=half}}
\makeatother
\usepackage{xcolor}
\IfFileExists{xurl.sty}{\usepackage{xurl}}{} % add URL line breaks if available
\IfFileExists{bookmark.sty}{\usepackage{bookmark}}{\usepackage{hyperref}}
\hypersetup{
  pdftitle={Project Report},
  pdfauthor={Caroline Andy, Vasili Fokaidis, Stella Li, Tessa Senders, Lily Wang},
  hidelinks,
  pdfcreator={LaTeX via pandoc}}
\urlstyle{same} % disable monospaced font for URLs
\usepackage[margin=1in]{geometry}
\usepackage{longtable,booktabs}
% Correct order of tables after \paragraph or \subparagraph
\usepackage{etoolbox}
\makeatletter
\patchcmd\longtable{\par}{\if@noskipsec\mbox{}\fi\par}{}{}
\makeatother
% Allow footnotes in longtable head/foot
\IfFileExists{footnotehyper.sty}{\usepackage{footnotehyper}}{\usepackage{footnote}}
\makesavenoteenv{longtable}
\usepackage{graphicx,grffile}
\makeatletter
\def\maxwidth{\ifdim\Gin@nat@width>\linewidth\linewidth\else\Gin@nat@width\fi}
\def\maxheight{\ifdim\Gin@nat@height>\textheight\textheight\else\Gin@nat@height\fi}
\makeatother
% Scale images if necessary, so that they will not overflow the page
% margins by default, and it is still possible to overwrite the defaults
% using explicit options in \includegraphics[width, height, ...]{}
\setkeys{Gin}{width=\maxwidth,height=\maxheight,keepaspectratio}
% Set default figure placement to htbp
\makeatletter
\def\fps@figure{htbp}
\makeatother
\setlength{\emergencystretch}{3em} % prevent overfull lines
\providecommand{\tightlist}{%
  \setlength{\itemsep}{0pt}\setlength{\parskip}{0pt}}
\setcounter{secnumdepth}{-\maxdimen} % remove section numbering

\title{Project Report}
\author{Caroline Andy, Vasili Fokaidis, Stella Li, Tessa Senders, Lily Wang}
\date{12/11/2020}

\begin{document}
\maketitle

\hypertarget{abstract}{%
\subsection{Abstract}\label{abstract}}

\textbf{Background:}~ Hate crimes are a growing public health threat in
the United States and are the highest priority of the FBI's civil rights
program (Miller et al., 2016). Existing research suggests that various
community-level socioeconomic factors, such as income inequality, are
associated with hate crime rate.

\textbf{Objectives:}\\
We aimed to analyze the association between community-level variables
and state-level rate of population-adjusted hate incident using data
reported to the Southern Poverty Law Center and analyzed in a 2017
FiveThrityEight article (Majumder, 2017).

\textbf{Methods:}\\
The data used for this project included state-level hate crime rates per
100,000 individuals in the population, as reported by the Southern
Poverty Law Center during the first weeks of November, 2016. Data
elements also included socioeconomic factors that are hypothesized to be
associated with hate crime.

The association between these socioeconomic factors and the hate crime
rate outcome were examined using multivariable linear regression
analysis. The outcome variable was first natural log transformed to
adhere to the linear regression assumption of normal distribution. After
identifying the statistically significant predictors of state-level hate
crime rate when controlling for all other covariates, we used automated
and criterion-based approaches to generate a parsimonious model that fit
our data well.

Multicollinearity of covariates, outliers, and variable interaction were
also tested for and considered in model development.

Finally, we assessed model goodness of fit and predictive performance
through model validation. All steps of model development and validation
were performed before and after removing the District of Columbia, an
outlier in the data.

\textbf{Results:}\\
Gini Index (an indicator of wealth inequality) and percent population
with a high school degree were both significant predictors of
state-level hate crime rate when controlling for all other covariates.
When these two predictors are the only predictors in the model they are
both positively associated with the outcome.

Of the continuous variables, percent non-citizen and percent non-white
were found to have a correlation coefficient of 0.753; median household
income and percentage of population with a high school degree had a
correlation coefficient of 0.651, both of which may suggest
multicollinearity. No statistically significant interactions were
identified between any variables in regression modeling.

Using automated and criterion-based approaches, and considering the body
of scientific evidence with regard to significant and practically
important socioeconomic factors associated with hate crimes, we
identified a regression model which optimizes parsimony and goodness of
fit. This model contains only the Gini Index and percent population with
a high school degree as predictors.

All steps of model development and validation were performed before and
after removing the District of Columbia, an outlier in the data. After
removing the District of Columbia, the Gini Index was no longer a
statistically significant predictor of hate crime rate; instead, the
percentage of the population with a high school degree was the only
significant variable, making the District of Columbia an influential
point.

\textbf{Conclusions:}\\
Based on the November 2016 Southern Poverty Law Center data, Gini Index
(an indicator of wealth inequality) and percent population with a high
school degree were both statistically significantly and independently
associated with state-level hate crime rate.

\hypertarget{introduction}{%
\subsection{Introduction}\label{introduction}}

As of 2020, the national rate of hate crimes in the United States is at
its highest level in over a decade {[}``FBI Incidents and Offenses,''
2019{]}. In the days following the 2016 US Presidential election, an
average of 90 hate crimes per day were reported to the Southern Poverty
Law Center (Miller et al., 2016).

Existing research suggests that community-level socioeconomic factors
such as racial breakdown, population density, level of educational
attainment, and economic considerations (median income, poverty level,
job availability) may be significant predictors of regional and
state-level rates of hate crimes (``FBI: Variables Affecting Crime,''
2012, Shively, 2005, Van Dyke et al., 2014). A 2017 FiveThirtyEight
article titled, ``Higher Rates Of Hate Crimes Are Tied To Income
Inequality,'' used 2016 FBI and Southern Poverty Law Center data to
assess the association between hate crime rate and select
community-level variables (Majumder, 2017).

For this project, we used this dataset to critically analyze this
research team's findings to identify state-level variables associated
with rates of hate crimes, and to generate a high performing predictive
model for population-adjusted hate incidents in the United States.

\hypertarget{methods}{%
\subsection{Methods}\label{methods}}

\hypertarget{data-exploration}{%
\paragraph{Data Exploration}\label{data-exploration}}

The data used for this project included state-level hate crime rates
(hate crimes per 100,000 population), as reported by the Southern
Poverty Law Center during the first weeks of November, 2016. Collected
state-level demographic variables include:

\begin{itemize}
\tightlist
\item
  Unemployment rate (high vs low) (as of 2016)
\item
  Urbanization (high vs low) (as of 2015)
\item
  Median household income (as of 2016)
\item
  Percent of residents with a high school degree (as of 2009)
\item
  Percent of residents who are non-citizens (as of 2015)
\item
  Income Gini coefficient (a measure of the extent to which the
  distribution of income among individuals within an economy deviates
  from a perfectly equal distribution; as of 2015)
\item
  Percent of residents who are non-White (as of 2015)
\end{itemize}

First, we investigated the extent of missing data in our dataset. Four
states--Hawaii, North Dakota, South Dakota, and Wyoming--did not report
hate crime rate data, and thus were excluded from subsequent analyses.
Three states--Maine, Mississippi, and South Dakota did not report their
percent of residents who were non-citizens. The District of Columbia was
included as a state for the purposes of these analyses.

Using these data, our goal was to construct our own multivariable linear
regression models to assess which of the collected variables, if any,
are statistically significantly associated (at a 0.05 significance
level) with population-adjusted hate incidents in the United States, to
altogether critically examine the article's findings. To do so, we first
generated descriptive statistics that we reported in a table which
includes the mean, median, range (min-max), interquartile range, and
count of missing entries for each numeric variable, and category percent
breakdowns and count of missing entries for categorical variables (Table
1). The variable Gini Index was particularly interesting as it indicated
income inequality in each state. The mean Gini Index value was 0.456
with a standard deviation of 0.021, and the range of these values was
from 0.419 to 0.532. Together, these values indicated great income
inequality in the overall dataset. Additionally, hate crime rate per
100,000 individuals had a mean and standard deviation of 0.315 and
0.150, respectively. The range of rates was from 0.067 to 1.522
indicating considerable variation in hate crime crimes per 100,000
individuals on a state-level. For categorical variables, the data showed
that majority of states had high unemployment and high urbanization at
51.1\% each.

In order to determine whether any data transformations would be
necessary we visualized the distribution of the outcome
(population-adjusted hate incidents per 100,000 population) by
generating an overlaid density and histogram plot (Figure 1).
Multivariable linear regression modeling operates under several
assumptions, which include residual homoscedasticity (constant variance)
and normality. Initial exploration of the distribution of the hate
crimes rate data showed a strong departure from standard normal
distribution with significant right skewness. Thus, we performed a Box
Cox transformation to isolate the `best' power transformation on the
hate crimes rate variable to achieve normal residuals.

To elucidate the existence of outliers and thus potential influential
points, we visualized the hate crime rate by state (Figure 3). We then
generated residuals vs fitted, normal Q-Q, scale-location, and residuals
vs leverage plots for a linear regression model that regressed the
transformed hate crime rate onto all possible covariates (Figure 4).
Specifically, we used the Residuals vs Leverage plot to identify any
outliers. Any points outside of Cook's Distance were considered
potential influential points. Any states that were deemed to be outliers
were included in the subsequent analyses, but the same analyses were
also run a second time on the dataset excluding the outliers for
comparison in order to determine if the outlier was indeed influencing
the results of our analyses.

\hypertarget{multicollinearity-and-interactions}{%
\paragraph{Multicollinearity and
Interactions}\label{multicollinearity-and-interactions}}

In order to investigate the existence of multicollinearity between each
of the continuous variables, we generated a correlation matrix. We
decided that any correlation coefficient of 0.6 and above may suggest
multicollinearity; thus, in these instances, at least one of those
correlated variables were dropped during subsequent model development.
Additionally, we calculated the variance inflation factors (VIFs) for
all variables, which quantify the degree of multicollinearity between
the given predictor all other remaining covariates.

Next, we plotted potential interactions between the continuous variables
and each categorical variable: urbanization and unemployment. These
plots were created once including outliers and once excluding outliers.
We then formally tested for significant interactions between any
variable pairs with intersecting lines. All interaction tests were
performed on datasets containing and not containing any observed
outliers.

\hypertarget{model-development-and-validation}{%
\paragraph{Model Development and
Validation}\label{model-development-and-validation}}

We began by performing two automated procedures for model variable
selection (forward and backward stepwise selection). We then used two
criterion based approaches (Mallow's Cp and adjusted r-squared). These
two criterion based approaches in R each generated the best performing
model which optimizes the given criterion for each possible number of
predictors. The first of these approaches used Mallow's Cp criterion,
which compares the predictive ability of model subsets to the full
model. To visualize these results, we generated a plot containing the Cp
criterion distribution for the top performing model for each number of
predictors (Figure 8). The next approach used adjusted r-squared which
favors models with a smaller SSE but also penalizes for additional
predictors. We generated a plot of the distribution of the adjusted R
squared statistic for the top performing model for each number of
included predictors (Figure 9).

SAY WE LOOKED AT DIAGNOSTICS FOR THE TWO COVARIATE MODEL WITH AND
WITHOUT DC

Using these results, we then identified the two best models which
maximize parsimony, interpretability and practical application, which
takes into account variables deemed to be significant and practically
important in existing literature. We then compared these two models
using an analysis of variance (a partial F-test for nested models).

We repeated this process of model selection on the dataset excluding any
observed outliers.

To assess the two models' predictive performance, we performed 5-fold
Cross Validation on each including and excluding outliers.

\hypertarget{results}{%
\subsection{Results}\label{results}}

\hypertarget{data-exploration-1}{%
\paragraph{Data Exploration}\label{data-exploration-1}}

The generated Box Cox transformation plot indicates that a lambda of 0
(i.e.~a natural logarithmic transformation of the hate crimes rate
outcome) would most closely approximate normal distribution (Figure 2).
Thus, we operated moving forward in model development using the natural
log of hate crimes rate as the outcome of interest.

Examining the Residuals vs Leverage plot generated from the regression
model including all possible covariates, we see that the District of
Columbia is clearly outside of the Cook's Distance line; thus we
concluded that this point is an outlier and a potential influential
point.

A table of basic descriptive statistics for the collected socioeconomic
variables is included in the appendix (Table 1). Continuous variables
show their mean, standard deviation, median, quartiles 1 and 3, their
minimum to maximum range of values, and number of missing values.
Categorical variables show their factor levels (i.e.~high and low),
percents of each factor level within the data set, and number of missing
values as well. Percent of population that are not U.S. citizens is the
only variable containing missing values (2 missing). FIX THIS PARAGRAPH!

\hypertarget{multicollinearity-and-interactions-1}{%
\paragraph{Multicollinearity and
Interactions}\label{multicollinearity-and-interactions-1}}

Out of the continuous variables, percent non-citizen and percent
non-white have a correlation coefficient of 0.753, and median household
income and percentage of population with a high school degree have a
correlation coefficient of 0.651, both of which may suggest
multicollinearity. All other correlation coefficients do not suggest
multi-collinearity (Table 2).

Use of variance inflation factors (VIFs) showed that the percent
population without a high school degree (3.895), percent non-citizen
(3.728), percent non-white (3.236) and median household income (3.108)
have the highest degrees of multicollinearity between all other
predictors. These values approach but do not exceed 4, the generally
accepted value which denotes the need for further investigation and/or
consideration of multicollinearity corrections.

For the dataset containing the District of Columbia, the interaction
plots suggest potential interaction between unemployment and median
household income, as well as between unemployment and percent population
of high school graduates, as seen by the differing directions in slope
(crossing lines). The urbanization interaction plots for these data
suggest that there may be potential interaction between urbanization and
each of the following variables: Gini Index, median household income,
percent non-citizen, and percent population of high school graduates.

For the dataset excluding the District of Columbia, the interaction
plots show potential interaction between unemployment and Gini Index as
well as between unemployment and percent population of high school
graduates. The urbanization interaction plots for these data suggest
potential interaction between urbanization and median household income
as well as between urbanization and percent population of high school
graduates.

When performing regression analyses for all of the above interactions
for each dataset, no interactions were found to be statistically
significant at a 0.05 significance level.

\hypertarget{model-development-and-validation-1}{%
\paragraph{Model Development and
Validation}\label{model-development-and-validation-1}}

We began our regression analyses by running linear regression models
containing all possible predictors for both the untransformed hate crime
rate and the natural log transformed hate crime rate. Our results
support the conclusions drawn in the FiveThirtyEight article: that the
Gini Index was the most significant independent predictor of state hate
crime rate when controlling for all other covariates and that the
percent high school graduates variable was the only other statistically
and independently significant variable (Majumder, 2017).

The model proposed during forward stepwise selection contained all
variables provided in the original dataset (Adjusted R-squared =
0.1849). The model generated through backward stepwise selection was
much more parsimonious and had a substantially higher adjusted R squared
(Adjusted R-squared = 0.2868). The only included predictors were the
Gini Index, and the percent of high school graduates, both of which were
significant. When the District of Columbia is removed from the dataset,
however, the Gini Index is no longer statistically significant at a 0.05
significance level, indicating that the District of Columbia is an
influential point. The adjusted R-squared of the two-predictor model
decreases as well when the District of Columbia is removed (from 0.2541
to 0.1185).

DESCRIBE ASSOCIATIONS BETWEEN COVARIATES AND OUTCOME (DIRECTION) AND
DISCUSS DIAGNOSTIC PLOTS

Both Mallow's Cp and the adjusted r-squared criterion confirmed that the
best model that includes two variables contains Gini Index and the
percent of high school graduates. Further, the plot for Mallow's Cp
showed that the model with only two predictors had the lowest Cp value
as compared to the best performing model of all other possible numbers
of predictors. The plot for the adjusted r-squared criterion showed that
the model with 3 predictors (Gini Index, the percent of high school
graduates, and unemployment) had the highest adjusted r-squared value.
However, the adjusted r-squared for this model was less than 6\% greater
than the adjusted r-squared for the model with only 2 covariates,
suggesting that their performance is not significantly different with
regard to this criterion. Further the partial F-test for nested models
shows that adding the unemployment variable does not significantly
improve the model. All of these results hold true when the District of
Columbia is removed from the dataset.

Through cross-validation, we determined that the aforementioned
two-predictor model also had better predictive performance than the
three-predictor model when including the District of Columbia. The CV
root mean squared error (RMSE) values are essentially equivalent
(0.5948853 for 2 covariates vs 0.6038494 for 3 covariates) and the CV
adjusted r-squared is slightly higher for the two-predictor model
(0.2943289 vs 0.2783783). When excluding the District of Columbia, the
two different models have virtually the same predictive performance.
Both have similar CV RMSE values and the CV adjusted r-squared for the
three-predictor model is only slightly higher than the two-predictor
model(0.1730294 vs 0.1530310). This indicates that the addition of
unemployment as a third predictor did not significantly add to model
performance whether the District of Columbia is included or not (Table
4). The two-predictor model, whether the District of Columbia is
included or not, performs well as a predictive model since the model's
RMSE and adjusted r-squared are very similar to the CV average RMSE and
adjusted r-squared.

{[}ADD FIGURE NUMBER AND THEN CITE SOMEWHERE ABOVE{]}

\hypertarget{discussionconclusion}{%
\subsection{Discussion/Conclusion}\label{discussionconclusion}}

After extensive analyses and modeling, we found that the optimal model
in terms of goodness of fit and predictive value contained the Gini
Index and percent population with a high school degree, with the latter
variable being a more significant predictor of hate crime in this
two-parameter model. However, these two predictors only accounted for
25.4\% of the variability in the data. This implies other variables not
in the data should be included in future studies of hate crime rates.
Such variables may include variables related to the politics of people
in the state or the percent of the population that are LGBTQ+ or percent
of the population that belongs to a religious minority as one study
showed that religious hate crimes are on the rise(``Incidents and
Offenses.'' FBI, FBI, 29 Oct.~2019).

The change in predictor significance of Gini Index and the adjusted
r-squared between the model including and excluding the District of
Columbia suggests it is indeed an influential point. The District of
Columbia has the highest hate crime rate and the highest income
inequality out of all the states.

ADD PARAGRAPH-WHY IS THIS STUDY IMPORTANT???

Unfortunately the data for hate crime rates was missing for four states.
These four states may have drastically different hate crime rates than
the rest of the states and could potentially change the significance of
the predictors and impact the predictive ability of our models,
therefore the results of this study may not be generalizable to the
entire US. Another limitation of our analyses is that we only tested for
interactions between the two ctageorical variables and each of the
continuous variables because these interactions are easier to understand
and stratify the data by. However, there could be interactions between
the continuous variables and even three-way interactions that explain
more of the variability in the data.

It is important to note that a natural logarithmic transformation of the
outcome was performed in order to satisfy the assumptions of linear
regression (residual homoscedasticity and normality) which impacts the
modeling results.

-why is this important? - limitations -assumptions - suggestions for
possible future studies

``Each figure and table should be of publishable quality and well
notated, i.e., labeled and/or captioned''

\hypertarget{figures-and-tables}{%
\subsection{Figures and Tables}\label{figures-and-tables}}

\includegraphics{project_report_files/figure-latex/unnamed-chunk-2-1.pdf}

\includegraphics{project_report_files/figure-latex/unnamed-chunk-3-1.pdf}

\includegraphics{project_report_files/figure-latex/unnamed-chunk-4-1.pdf}

\includegraphics{project_report_files/figure-latex/unnamed-chunk-5-1.pdf}
\includegraphics{project_report_files/figure-latex/unnamed-chunk-5-2.pdf}

\begin{longtable}[]{@{}ll@{}}
\caption{Descriptive Statistics of States Reporting Hate
Crimes}\tabularnewline
\toprule
& Overall (N=47)\tabularnewline
\midrule
\endfirsthead
\toprule
& Overall (N=47)\tabularnewline
\midrule
\endhead
\textbf{Unemployment} &\tabularnewline
~~~high & 24 (51.1\%)\tabularnewline
~~~low & 23 (48.9\%)\tabularnewline
~~~Missing & 0\tabularnewline
\textbf{Urbanization} &\tabularnewline
~~~low & 23 (48.9\%)\tabularnewline
~~~high & 24 (51.1\%)\tabularnewline
~~~Missing & 0\tabularnewline
\textbf{Median Household Income} &\tabularnewline
~~~Mean (SD) & 54802.298 (9255.117)\tabularnewline
~~~Median (Q1, Q3) & 54310.000 (47629.500, 60597.500)\tabularnewline
~~~Min - Max & 35521.000 - 76165.000\tabularnewline
~~~Missing & 0\tabularnewline
\textbf{\% Adults \textgreater25yrs With HS Degree} &\tabularnewline
~~~Mean (SD) & 0.866 (0.034)\tabularnewline
~~~Median (Q1, Q3) & 0.871 (0.839, 0.895)\tabularnewline
~~~Min - Max & 0.799 - 0.915\tabularnewline
~~~Missing & 0\tabularnewline
\textbf{\% of Population Not U.S. Citizens} &\tabularnewline
~~~Mean (SD) & 0.055 (0.031)\tabularnewline
~~~Median (Q1, Q3) & 0.050 (0.030, 0.080)\tabularnewline
~~~Min - Max & 0.010 - 0.130\tabularnewline
~~~Missing & 2\tabularnewline
\textbf{Gini Index} &\tabularnewline
~~~Mean (SD) & 0.456 (0.021)\tabularnewline
~~~Median (Q1, Q3) & 0.455 (0.441, 0.468)\tabularnewline
~~~Min - Max & 0.419 - 0.532\tabularnewline
~~~Missing & 0\tabularnewline
\textbf{\% of Population Not White} &\tabularnewline
~~~Mean (SD) & 0.315 (0.150)\tabularnewline
~~~Median (Q1, Q3) & 0.300 (0.205, 0.420)\tabularnewline
~~~Min - Max & 0.060 - 0.630\tabularnewline
~~~Missing & 0\tabularnewline
\textbf{Hate Crime Rate Per 100k} &\tabularnewline
~~~Mean (SD) & 0.304 (0.253)\tabularnewline
~~~Median (Q1, Q3) & 0.226 (0.143, 0.357)\tabularnewline
~~~Min - Max & 0.067 - 1.522\tabularnewline
~~~Missing & 0\tabularnewline
\bottomrule
\end{longtable}

\includegraphics{project_report_files/figure-latex/unnamed-chunk-7-1.pdf}

\begin{longtable}[]{@{}lr@{}}
\caption{VIF Values}\tabularnewline
\toprule
& VIF\tabularnewline
\midrule
\endfirsthead
\toprule
& VIF\tabularnewline
\midrule
\endhead
unemployment & 1.426\tabularnewline
urbanization & 1.983\tabularnewline
median\_household\_income & 3.108\tabularnewline
perc\_pop\_hs & 3.895\tabularnewline
perc\_non\_citizen & 3.728\tabularnewline
gini\_index & 1.845\tabularnewline
perc\_non\_white & 3.236\tabularnewline
\bottomrule
\end{longtable}

\includegraphics{project_report_files/figure-latex/unnamed-chunk-9-1.pdf}

\includegraphics{project_report_files/figure-latex/unnamed-chunk-10-1.pdf}

\includegraphics{project_report_files/figure-latex/unnamed-chunk-11-1.pdf}

\begin{longtable}[]{@{}lrrrr@{}}
\caption{Comparison of Adjusted R\^{}2 and RMSE for All
Models}\tabularnewline
\toprule
& Model adjusted R\^{}2 & Model RMSE & CV adjusted R\^{}2 & CV
RMSE\tabularnewline
\midrule
\endfirsthead
\toprule
& Model adjusted R\^{}2 & Model RMSE & CV adjusted R\^{}2 & CV
RMSE\tabularnewline
\midrule
\endhead
Two predictors with DC & 0.2541 & 0.5417445 & 0.2943289 &
0.5948853\tabularnewline
Three predictors with DC & 0.2571 & 0.5341956 & 0.2783783 &
0.6038494\tabularnewline
Two predictors without DC & 0.1185 & 0.5367246 & 0.1530310 &
0.5554347\tabularnewline
Three predictors without DC & 0.1250 & 0.5281813 & 0.1730294 &
0.5600140\tabularnewline
\bottomrule
\end{longtable}

\hypertarget{references}{%
\subsection{References}\label{references}}

\begin{itemize}
\tightlist
\item
  ``Incidents and Offenses.'' FBI, FBI, 29 Oct.~2019,
  ucr.fbi.gov/hate-crime/2019/topic-pages/incidents-and-offenses.
\item
  Majumder, M. ``Higher Rates Of Hate Crimes Are Tied To Income
  Inequality.'' FiveThirtyEight, FiveThirtyEight, 23 Jan.~2017,
  fivethirtyeight.com/features/higher-rates-of-hate-crimes-are-tied-to-income-inequality/.
\item
  Miller, C., Werner-Winslow, A. ``Ten Days After: Harassment and
  Intimidation in the Aftermath of the Election.'' Southern Poverty Law
  Center, 29 Nov.~2016,
  www.splcenter.org/20161129/ten-days-after-harassment-and-intimidation-aftermath-election.\\
\item
  Shively, Michael. Abt Associates Inc, 2005, Study of Literature and
  Legislation on Hate Crime in America.
\item
  Van Dyke, N., and Tester, G. ``Dangerous Climates.'' Journal of
  Contemporary Criminal Justice, vol.~30, no. 3, 2014, pp.~290--309.,
  \url{doi:10.1177/1043986214536666}.
\item
  ``Variables Affecting Crime.'' FBI, FBI, 5 Nov.~2012,
  ucr.fbi.gov/hate-crime/2011/resources/variables-affecting-crime.
\end{itemize}

\end{document}
